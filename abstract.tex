\sectioncentered*{Реферат}
\thispagestyle{empty}

% Зачем: чтобы можно было вывести общее число страниц.
% Добавляется единица, поскольку последняя страница -- ведомость.
\FPeval{\totalpages}{round(\getpagerefnumber{LastPage} + 1, 0)}

\MakeUppercase{Программное средство учёта персонального бюджета для платформы Android с применением Unit-\-тес\-ти\-ро\-ва\-ния и Mock технологии}: дипломный проект / М. В. Снитовец.~-- Минск : БГУИР, 2018,~-- п.з.~-- \totalpages~с., чертежей (плакатов)~-- 6 л. формата А1.
\vspace{\baselineskip}

Цель настоящего дипломного проекта состоит в  разработке программного средства, которое упрощает и ускоряет основные задачи учёта персонального бюджета, для платформы \andro.

В процессе анализа предметной области были выделены основные аспекты процесса учёта персонального бюджета.

Было проведено их исследование и моделирование.
Кроме того, рассмотрены существующие средства схожей функциональности.

Выработаны функциональные и нефункциональные требования.

Была разработана архитектура программной системы, для каждой ее составной части было проведено разграничение реализуемых задач проектирование, уточнение используемых технологий и собственно разработка.
Были выбраны наиболее современные средства разработки, широко применяемые в индустрии.

Полученные в ходе технико-экономического обоснования результаты о прибыли для разработчика, пользователя, уровень рентабельности, а также экономический эффект доказывают целесообразность разработки про\-екта.
