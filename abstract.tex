\sectioncentered*{Реферат}
\thispagestyle{empty}

% Зачем: чтобы можно было вывести общее число страниц.
% Добавляется единица, поскольку последняя страница -- ведомость.
\FPeval{\totalpages}{round(\getpagerefnumber{LastPage} + 1, 0)}

\begin{center}
	Пояснительная записка \num{\totalpages}~с., 
	\num{\totfig{}}~рис., 
	\num{\tottab{}}~табл., 
	\num{\toteq{}}~формул, 
	\num{\totref{}}~источников.
	\MakeUppercase{Программное средство, мобильное приложение, бюджет, учёт доходов и расходов, статистика}
\end{center}

Цель настоящего дипломного проекта состоит в разработке программного средства, предназначенного для простой и эффективной автоматизации задач учёта персонального бюджета.


В процессе анализа предметной области были выделены основные аспекты процесса учёта персонального бюджета.

Было проведено их исследование и моделирование.
Кроме того, рассмотрены существующие средства схожей функциональности (так называемые частичные аналоги).

Выработаны функциональные и нефункциональные требования.

Была разработана архитектура программной системы, для каждой ее составной части было проведено разграничение реализуемых задач проектирование, уточнение используемых технологий и собственно разработка.
Были выбраны наиболее современные средства разработки, широко применяемые в индустрии.

Полученные в ходе технико-экономического обоснования результаты о прибыли для разработчика, пользователя, уровень рентабельности, а также экономический эффект доказывают целесообразность разработки про\-екта.
