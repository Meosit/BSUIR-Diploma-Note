\subsection{Аналитический обзор литературных источников}
\label{sec:analysis:literature}

Далее приводится анализ сведений, которые влияют на формулирование требований, дальнейшее проектирование и разработку программного средства.

\subsubsection{} Обзор основных составляющих персонального бюджета
\label{sec:analysis:literature:components}

Бюджет -- это документ (электронный или бумажный), в котором регулярно, наглядно и детально отображаются все статьи доходов и расходов за конкретный период времени, другими словами -- все источники притока средств, все траты, а также какие-либо индивидуальные правила распоряжения финансами и личный финансовый план н будущее~\cite{budget_blog}.

Ведение персонального бюджета включает в себя несколько базовых частей, которые со временем дополняются другими, приобретая черты более сложной системы.
Основными компонентами ведения личного бюджета являются:
\begin{itemize}
    \item учёт доходов и расходов;
    \item оптимизация расходов;
    \item планирование доходов и расходов.
\end{itemize}

Стоит отметить, что каждый последующий компонент является логическим продолжением предыдущего.

В прикладном прогнозировании (прогнозировании) нередко возникают ситуации, когда математическое моделирование, основанное на использовании точных законов оказывается затруднительны, но в распоряжении исследователей оказывается выборка прецедентов -- результатов наблюдений исследуемого процесса или явления, включающих значения прогнозируемой величины~\cite{prediction_basics}.
Так как каждый человек распоряжается деньгами уникальным образом, то получение точных законов по планированию расходов практически невозможно.
Поэтому \emph{учёт доходов и расходов} является необходимым и основным компонентом учёта персонального бюджета.
Данный компонент отвечает за получение исторических и текущих данных о расходах, которые можно будет использовать в планировании и оптимизации.
Учёт доходов необходим для того чтобы точно определить конкретные источники доходов, их регулярность и величину.
В то же время учёт расходов более важен, так как позволяет определить конкретные категории денежных потоков, какие из них являются необходимыми, а какие можно ограничить или убрать вовсе.

\emph{Оптимизация расходов} подразумевает рациональное использование средств.
Основной целью данного этапа является выявление наиболее убыточных категорий, отсутствие трат на которые незначительно изменит общий уровень жизни. Также в рамках данного этапа необходимо выделить \emph{обязательные} и \emph{необязательные} категории расходов.
Например, категорию <<коммунальные услуги>>  можно отнести к первому типу, а <<кино>> -- ко второму.
Оптимизация расходов в основном направлена на необязательные категории расходов так как они зачастую непостоянны и напрямую зависят от желаний и самоконтроля человека.

Расходы на обязательные категории трудно сократить, так как зачастую человек не может прямым образом влиять на их размер: ставки на электроэнергию и водоснабжение устанавливаются внешними структурами.
При этом данный тип категорий легко поддается планированию так как эти суммы достаточно постоянны.

Любое планирование -- это существенная составляющая успеха в любой сфере жизни,  так как позволяет не только разделить процесс достижения цели на несколько важных этапов, но и увидеть новые возможности.

\emph{Финансовое планирование} -- выбор целей по реальности их достижения с имеющимися финансовыми ресурсами в зависимости от внешних условий и согласование будущих финансовых потоков, выражается в составлении и контроле над выполнением планов формирования доходов и расходов, учитывающих текущее финансовое состояние, выраженные в денежном эквиваленте цели и средства их достижения~\cite{finance_planning}.

Финансовые (и любые другие) планы подразделяются, как правило, на краткосрочные (до 1 года), среднесрочные (от 1 года до 3 лет) и долгосрочные (от 3 лет и более).
Для наилучшего результата рекомендуется планировать начиная с краткосрочных планов.
Во-первых, достижение целей -- это процесс поэтапный, в котором реализация среднесрочных или долгосрочных планов может зависеть от краткосрочных или среднесрочных.
И, во-вторых, всегда существует определенный рубеж, преодолеть который в конкретный момент человек не может.
К тому же, имеют место быть некоторые обстоятельства, повлиять на которые достаточно сложно.
К таким обстоятельствам можно отнести инфляцию, внезапное сокращение на работе, непредвиденные необходимые траты и прочее.
Чтобы иметь возможность быть готовым, если не ко всему, то ко многому, требуется иметь чёткое представление о том, что делать в той или иной ситуации, а также разработать свою стратегию по достижению целей.
Всё это и включает в себя финансовое планирование.
Лучшим временем для приведения бюджета в порядок и планирования является начало года, однако это не является обязательным требованием.

Полученная информация поможет определить, на какие аспекты учёта персонального бюджета следует обратить повышенное внимание во время формирования требований к проектируемому средству.

\subsubsection{} Обзор подходов по учёту доходов и расходов
\label{sec:analysis:literature:tracking}

Как было упомянуто в подразделе~\ref{sec:analysis:literature:components}, учёт доходов и расходов является базовым компонентом для качественного учёта персонального бюджета.
Существует множество разнообразных средств для реализации этой цели.
Независимо от того, какое средство будет задействовано, все они рассчитывают на точность введенных данных.

\emph{Бумажный метод} учёта доходов и расходов существует уже десятки лет и является одним из первых подходов для достижения финансовой стабильности.

В самом простом своём появлении данный метод требует только простого блокнота и ручки, куда записывают все траты по мере их поступления (см. рисунок~\ref{fig:analysis:literature:paper_log}), либо в конце дня по наличию чеков оплаты.
Также существуют специальные бумажные издания с напечатанными таблицами, которые заполняют информацией о доходах.

\begin{figure}[H]
    \centering
    \includegraphics[scale=0.52]{1_1_2_paper_log.png}
    \caption{Пример ручного учёта расходов в блокноте}
    \label{fig:analysis:literature:paper_log}
\end{figure}

Однако бумажный метод является относительно устаревшим и имеет несколько недостатков, таких как необходимость постоянно носить упомянутый блокнот с собой, низкая скорость записи и изменения транзакций, необходимость подсчитывать суммы вручную.

С развитием информационных технологий появились различные решения с использованием вычислительной техники.
Одним из первых таких решений является использование программного обеспечения по работе с электронными таблицами, такими как \emph{Microsoft Excel} или \emph{Google Spreadsheets}.

Подобные программные решения упрощают занесение и редактирование новой информации, также имеются большие возможности по использованию встроенных формул и построения графиков.
Среди недостатков стоит выделить тот факт, что подобные системы в основном рассчитаны на использование с персонального компьютера, что сильно затрудняет своевременный ввод информации о транзакциях из-за того компьютер зачастую недоступен в момент совершения операций с деньгами.

Также существуют специализированные программные решения по учёту доходов и расходов для различных платформ.
Среди них выделяют:
\begin{itemize}
    \item настольные приложения;
    \item мобильные приложения;
    \item веб-приложения.
\end{itemize}

\emph{Настольное приложение} -- программное средство, которое запускается локально на компьютере пользователя.
Такие приложения имеют богатую функциональность для решения задач по учёту доходов и расходов, планированию и отображению различных графиков и диаграмм.
Однако, как и ПО для работы с электронными таблицами, данный тип приложений зачастую не находится в быстром доступе.
Данного недостатка нет у \emph{мобильных приложений} -- в современном мире почти всегда люди имеют под рукой мобильное устройство.
Также появляется возможность использования мгновенных оповещений~\cite{desktop_mobile_differences} и интеграциями с другими особенностями мобильных устройств, таких.
Но для таких приложений актуальны ограничения платформ, на которых они запускаются, такие как меньшие размеры экранов, более медленные процессоры, ограниченное энергопотребление.
\emph{Веб-приложения}, в отличие от настольных и мобильных, работают на удаленном аппаратном обеспечении и поставляются пользователю через браузер~\cite{web_based_vs_desktop}.
Это позволяет использовать его как с мобильного устройства, так и с персонального компьютера.
Несмотря на то, задержки при передаче, особенно при нестабильном соединении, сама потребность в постоянном интернет соединении могут значительно снизить удобство пользования приложением.
Кроме того, размер загружаемого при каждом запуске кода и ресурсов может значительно увеличить траты пользователя, особенно если он использует дорогое мобильное подключение к интернету.

\subsubsection{} Обзор методик планирования бюджета
\label{sec:analysis:literature:optimization}

% TODO: Add expense optimization overview

\subsubsection{} Анализ операционной системы Android в качестве целевой платформы
\label{sec:analysis:literature:android}

% TODO: Add android analysis (statistics, drawbacks and advantages)