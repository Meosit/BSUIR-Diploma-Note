\subsection{Аналитический обзор литературных источников}
\label{sec:analysis:literature}

Далее приводится анализ сведений, которые влияют на формулирование требований, дальнейшее проектирование и разработку программного средства.

\subsubsection{} Обзор основных составляющих персонального бюджета
\label{sec:analysis:literature:components}

Бюджет – это документ (электронный или бумажный), в котором регулярно, наглядно и детально отображаются все статьи доходов и расходов за конкретный период времени, другими словами -- все источники притока средств, все траты, а также какие-либо индивидуальные правила распоряжения финансами и личный финансовый план н будущее.~\cite{budget_blog} 

Ведение персонального бюджета включает в себя несколько базовых частей, которые со временем дополняются другими, приобретая черты более сложной системы. 
Основными компонентами ведения личного бюджета являются: 
\begin{itemize}
    \item учёт доходов и расходов;
    \item оптимизация расходов;
    \item планирование доходов и расходов.
\end{itemize}
Стоит отметить, что каждый последующий компонент является логическим продолжением предыдущего.


% TODO: Add budget components overview

\subsubsection{} Обзор подходов по учёту доходов и расходов
\label{sec:analysis:literature:tracking}

Как было упомянуто в подразделе~\ref{sec:analysis:literature:components}, учёт доходов и расходов является базовым компонентом для качественного учёта персонального бюджета. 

С появлением новых технологий изменялись и подходы по учёту доходов и расходов 

% TODO: Add income/expense tracking overview

\subsubsection{} Обзор методик оптимизации расходов
\label{sec:analysis:literature:optimization}

% TODO: Add expense optimization overview

\subsubsection{} Анализ операционной системы Android в качестве целевой платформы
\label{sec:analysis:literature:android}

% TODO: Add android analysis (statistics, drawbacks and advantages)