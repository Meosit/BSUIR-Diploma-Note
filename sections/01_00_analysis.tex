\section{Анализ литературных источников, прототипов и формирование требований к проектируемому программному средству}
\label{sec:analysis}

Успех программного проекта во многом определяется на этапе подготовки, которая проводится с учетом всех особенностей проекта.

Первое предварительное условие, которое нужно выполнить перед конструированием, -- ясное формулирование проблемы, которую система должна решать.
Снижение риска является одной из основных целей подготовки: качественное планирование позволяет исключить главные аспекты риска на самых ранних стадиях работы.

Главный факторы риска в создании программного обеспечения — неудачная выработка требований.
Требования подробно описывают, что должна делать программная система.
Внимание к требованиям помогает свести к минимуму изменения системы после начала разработки~\cite{code_complete}.

Перед формулированием требований необходимо изучить ряд вопросов, которые напрямую влияют на все дальнейшие этапы разработки.
В частности, необходимо рассмотреть основные компоненты учёта персонального бюджета, методики и подходы оптимизации расходов, преимущества и недостатки целевой платформы.
По результатам анализа можно будет составить техническое задание к проектируемому программному средству, которое станет основой для составления функциональных требований.

\subsection{Аналитический обзор литературных источников}
\label{sec:analysis:literature}

Далее приводится анализ сведений, которые влияют на формулирование требований, дальнейшее проектирование и разработку программного средства.

\subsubsection{} Обзор основных составляющих персонального бюджета
\label{sec:analysis:literature:components}

% TODO: Add budget components overview

\subsubsection{} Обзор подходов по учёту доходов и расходов
\label{sec:analysis:literature:tracking}

% TODO: Add income/expense tracking overview

\subsubsection{} Обзор методик оптимизации расходов
\label{sec:analysis:literature:optimization}

% TODO: Add expense optimization overview

\subsubsection{} Анализ операционной системы Android в качестве целевой платформы
\label{sec:analysis:literature:android}

% TODO: Add android analysis (statistics, drawbacks and advantages)

\subsection{Обзор существующих аналогов}
\label{sec:analysis:analogues}

% TODO: Add analogues overview


\subsection{Требования к проектируемому программному средству}
\label{sec:analysis:specification}

По результатам изучения предметной области, анализа литературных источников и обзора существующих систем-аналогов сформулируем требования к проектируемому программному средству.

\subsubsection{} Назначение проекта
\label{sec:analysis:specification:purpose}

Назначением проекта является разработка программного средства, автоматизирующего основные задачи по учёту персонального бюджета.

\subsubsection{} Основные функции
\label{sec:analysis:specification:functions}

Программное средство должно поддерживать следующие основные фун\-к\-ции:

\begin{itemize}
    \item управление категориями расходов и доходов;
    \item управление счетами;
    \item управление транзакциями;
    \item отображение списка транзакций по дням;
    \item расчёт и отображение статистики по категориям;
    \item расчёт и отображение агрегатных значений по счетам.
\end{itemize}

\subsubsection{} Требования к входным данным
\label{sec:analysis:specification:inputs}

Входные данные для программного средства должны быть представлены в виде вводимого пользователем с помощью клавиатуры текста и выбора доступных опций пользовательского интерфейса.

Должны быть реализованы проверки вводимых данных на корректность с отображением информации об ошибках в случае их некорректности.

\subsubsection{} Требования к выходным данным
\label{sec:analysis:specification:outputs}

Выходные данные программного средства должны быть представлены посредством отображения информации с помощью различных элементов пользовательского интерфейса.

\subsubsection{} Требования к составу и параметрам технических и программных средств
\label{sec:analysis:specification:minimal_requirements}

Программное средство должно функционировать на устройстве со следующими минимальными характеристиками:

\begin{itemize}
    \item тактовая частота процессора от 2 ГГц и более;
    \item оперативная память 2 Гб и более;
    \item диагональ дисплея 4.5 дюймов и более;
    \item версия операционной системы \andro 4.4 и выше.
\end{itemize}

% TODO: Refactor and enhance minimal_requirements section

\subsubsection{} Требования к информационной и программной совместимости
\label{sec:analysis:specification:compatibility}

% TODO: Add compatibility requirements
