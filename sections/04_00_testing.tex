\section{Тестирование программного средства}
\label{sec:testing}

Тестирование программного обеспечения -- процесс анализа программного средства и сопутствующей документации с целью выявления дефектов и повышения качества продукта~\cite{kulikov_testing}.
Вот уже несколько десятков лет его стабильно включают в планы разработки как одна из основных работ, причем выполняемая практически на всех этапах проектов.

Тестирование можно классифицировать по очень большому количеству признаков. Основные виды классификации включают следующие~\cite{kulikov_testing}:

\begin{enumerate}
    \item по запуску кода на исполнение:
    \begin{enumerate}
        \item статическое тестирование -- без запуска программного средства;
        \item динамическое тестирование -- с запуском;
    \end{enumerate}
    \item по степени автоматизации:
    \begin{enumerate}
        \item ручное тестирование -- тестовые случаи выполняет человек;
        \item автоматизированное тестирование -- тестовые случаи частично или полностью выполняет специальное инструментальное средство;
    \end{enumerate}
    \item по принципам работы с приложением:
    \begin{enumerate}
        \item позитивное тестирование -- все действия с приложением выполняются строго в соответствии с требованиями без недопустимых действий или некорректных данных;
        \item негативное тестирование -- проверяется способность приложения продолжать работу в критических ситуациях недопустимых действий или данных.
    \end{enumerate}
\end{enumerate}

Одним из самых распространенных типов статического автоматизированного тестирования является Unit-тестирование -- подход, позволяющий проверить на корректность единицы исходного кода, наборы из одного или более программных модулей вместе с соответствующими управляющими данными, процедурами использования и обработки.

Из преимуществ модульного тестирования стоит выделить:

\begin{itemize}
    \item Возможность проведения рефакторинга с уверенностью в том, что модуль по-прежнему работает корректно, при условии, что все тесты пройдены (регрессионное тестирование).
    Это поощряет программистов к изменениям кода с целью его оптимизации и улучшения, поскольку достаточно легко проверить, что код работает и после изменений.
    \item Упрощение интеграции в силу того, что существует возможность протестировать отдельные модули программного средства, а затем его в целом.
    \item Модульные тесты можно рассматривать как <<живой документ>> для тестируемого класса.
    Программисты, которые не знают, как использовать данный класс, могут использовать Unit-тест в качестве примера.
    \item Подход позволяет выявить и устранить излишние зависимости между модулями проекта из-за того, что каждый тест нацелен отдельный модуль программного средства и дополнительные зависимости затрудняют написание их реализацию.
\end{itemize}

Тем не менее, модульное тестирование не применимо в следующих случаях:

\begin{itemize}
    \item Результат известен лишь приблизительно.
    Например, в математическом моделировании.
    Данный вид тестирования подразумевает, что входные и выходные данные модуля будут однозначно и точно определены.
    \item Необходимость тестирования взаимодействия модулей.
    При выполнении Unit-тестов происходит тестирование каждого из модулей по отдельности, поэтому ошибки интеграции, системного уровня, функций, исполняемых в нескольких модулях, не будут определены.
\end{itemize}

В рамках дипломного проектирования было выбрано модульное тестирование в результате которого можно определить, что логика предметной области приложения работает в соответствии с требованиями, установленными в начале разработки.

Для обеспечения изолированности модуля в рамках теста, требуется убрать его их зависимости на внешние модули, такие как Storage, описанные в подразделе~\ref{sec:design:business}.
Для этого необходимо использование Mock-объектов.
Mock-объект представляет собой конкретную фиктивную реализацию интерфейса, предназначенную исключительно для тестирования взаимодействия и относительно которого высказывается утверждение.
Также, при использовании данного объекта, он запоминает все совершенные с ним действия, такие как вызванные методы и переданные в них значения параметров.
Позже собранная информация помогает проверить правильное взаимодействие данного модуля с другими.

Некоторые результаты тестирования представлены в таблице~\ref{tbl:testing:unit}.

По результатам тестирования были исправлены найденные ошибки в предметной логике разработанного программного средства.

% Зачем: свой счетчик для нумерации тестов.
\newcounter{testnumber}
\newcommand\testnumber{\stepcounter{testnumber}\arabic{testnumber}}

\begin{landscape}
    \begin{longtable}{|>{\centering}m{0.21\textwidth}
                      |m{0.7\textwidth}
                      |m{0.34\textwidth}
                      |>{\centering\arraybackslash}m{0.16\textwidth}|}
        \caption{Тестовые случаи Unit-тестирования}
        \label{tbl:testing:unit}\\

        \hline
        \begin{minipage}{1\linewidth}
        \centering Тестовый случай
        \end{minipage} &
        \begin{minipage}{1\linewidth}
        \centering Предварительные условия
        \end{minipage} &
        \begin{minipage}{1\linewidth}
        \centering Ожидаемые результаты
        \end{minipage} &
        \centering\arraybackslash Тестовый случай пройден? \endfirsthead
        \caption*{Продолжение таблицы~\ref{tbl:testing:unit}}\\\hline
        \centering 1 & \centering 2 & \centering 3 & \centering\arraybackslash 4 \\\hline \endhead

        \hline
        \centering 1 & \centering 2 & \centering 3 & \centering\arraybackslash 4 \\
        \hline
        % КОНЕЦ ЗАГОЛОВКА

        % Тестовый случай
        \testnumber. Удаление счёта
        & % Предусловия
        \begin{minipage}[t]{1\linewidth}
            \begin{itemize}
                \item Mock-объект для WalletStorage c поведением по умолчанию.
                \item Mock-объект для TransactionStorage c поведением по умолчанию.
                \item Метод удаления всех транзакций счёта из БД возвращает <<5>>.
                \item Реализация WalletService c зависимостями на созданные Mock-объекты.
                \item Идентификатор счёта для удаления равен <<12>>.
            \end{itemize}
        \end{minipage}
        & % Ожидаемые результаты
        \begin{minipage}[t]{1\linewidth}
            \begin{itemize}
                \item Метод для удаления всех транзакций счёта из БД был вызван один раз.
                \item Вызван метод удаления счета из БД с идентификатором <<12>>.
                \item Возвращенное из метода значение равно <<5>>.
            \end{itemize}
        \end{minipage}
        & % Тестовый случай пройден?
        Да
        \\
        & & & \\
        \hline
        % КОНЕЦ ТЕСТА

        % Тестовый случай
        \testnumber. Удаление категории с удалением зависимых транзакций
        & % Предусловия
        \begin{minipage}[t]{1\linewidth}
            \begin{itemize}
                \item Mock-объект для CategoryStorage c поведением по умолчанию.
                \item Mock-объект для TransactionStorage c поведением по умолчанию.
                \item Метод удаления всех транзакций категории из БД возвращает <<15>>.
                \item Реализация CategoryService c зависимостями на созданные Mock-объекты.
                \item Идентификатор категории для удаления равен <<12>>.
            \end{itemize}
        \end{minipage}
        & % Ожидаемые результаты
        \begin{minipage}[t]{1\linewidth}
            \begin{itemize}
                \item Метод для удаления всех транзакций категории из БД был вызван один раз.
                \item Вызван метод удаления категории из БД с идентификатором <<12>>.
                \item Возвращенное из метода значение равно <<15>>.
            \end{itemize}
        \end{minipage}
        & % Тестовый случай пройден?
        Да
        \\
        & & & \\
        %\hline
        % КОНЕЦ ТЕСТА

        % Тестовый случай
        \testnumber. Удаление категории с переносом транзакций
        & % Предусловия
        \begin{minipage}[t]{1\linewidth}
            \begin{itemize}
                \item Mock-объект для CategoryStorage c поведением по умолчанию.
                \item Mock-объект для TransactionStorage c поведением по умолчанию.
                \item Метод удаления всех транзакций категории из БД возвращает <<11>>.
                \item Реализация CategoryService c зависимостями на созданные Mock-объекты.
                \item Идентификатор категории для удаления равен <<12>>.
                \item Идентификатор категории для переноса равен <<66>>.
            \end{itemize}
        \end{minipage}
        & % Ожидаемые результаты
        \begin{minipage}[t]{1\linewidth}
            \begin{itemize}
                \item Метод для обновления всех транзакций категории из БД был вызван один раз.
                Старый идентификатор равен <<12>>, новый -- <<66>>.
                \item Вызван метод удаления категории из БД с идентификатором <<12>>.
                \item Возвращенное из метода значение равно <<11>>.
            \end{itemize}
        \end{minipage}
        & % Тестовый случай пройден?
        Да
        \\
        & & & \\
        \hline
        % КОНЕЦ ТЕСТА

        % Тестовый случай
        \testnumber. Создание транзакции для категории расходов
        & % Предусловия
        \begin{minipage}[t]{1\linewidth}
            \begin{itemize}
                \item Mock-объект для TransactionStorage c поведением по умолчанию.
                \item Mock-объект для WalletStorage c поведением по умолчанию.
                \item Реализация TransactionService c зависимостями на созданные Mock-объекты.
                \item Транзакция на сумму <<10,5>> и категорией типа <<EXPENSE>>.
                \item Счёт транзакции и балансом <<25>>.
            \end{itemize}
        \end{minipage}
        & % Ожидаемые результаты
        \begin{minipage}[t]{1\linewidth}
            \begin{itemize}
                \item Был вызван метод создания транзакции на сумму <<10.5>>.
                \item Был вызван обновления баланса счёта с <<25>> на <<14.5>>.
            \end{itemize}
        \end{minipage}
        & % Тестовый случай пройден?
        Да
        \\
        & & & \\
        %\hline
        % КОНЕЦ ТЕСТА

        % Тестовый случай
        \testnumber. Удаление транзакции из категории расходов
        & % Предусловия
        \begin{minipage}[t]{1\linewidth}
            \begin{itemize}
                \item Mock-объект для TransactionStorage c поведением по умолчанию.
                \item Mock-объект для WalletStorage c поведением по умолчанию.
                \item Реализация TransactionService c зависимостями на созданные Mock-объекты.
                \item Транзакция с идентификатором <<12>> на сумму <<9>> и категорией типа <<EXPENSE>>.
                \item Счёт транзакции и балансом <<21>>.
            \end{itemize}
        \end{minipage}
        & % Ожидаемые результаты
        \begin{minipage}[t]{1\linewidth}
            \begin{itemize}
                \item Был вызван метод удаления транзакции с идентификатором <<12>>.
                \item Был вызван обновления баланса счёта с <<21>> на <<30>>.
            \end{itemize}
        \end{minipage}
        & % Тестовый случай пройден?
        Да
        \\
        & & & \\
        \hline
        % КОНЕЦ ТЕСТА

        % Тестовый случай
        \testnumber. Изменение счёта транзакции
        & % Предусловия
        \begin{minipage}[t]{1\linewidth}
            \begin{itemize}
                \item Mock-объект для TransactionStorage c поведением по умолчанию.
                \item Mock-объект для WalletStorage c поведением по умолчанию.
                \item Реализация TransactionService c зависимостями на созданные Mock-объекты.
                \item Объект транзакции с идентификатором <<12>> на сумму <<9>> и категорией типа <<EXPENSE>>.
                \item Счёт транзакции до изменения с балансом <<21>>.
                \item Счёт транзакции после изменения с балансом <<41>>.
            \end{itemize}
        \end{minipage}
        & % Ожидаемые результаты
        \begin{minipage}[t]{1\linewidth}
            \begin{itemize}
                \item Был вызван метод изменения счёта транзакции с идентификатором <<12>>.
                \item Вызван метод обновления баланса счёта до изменения с <<21>> на <<30>>.
                \item Был вызван метод обновления баланса счёта после изменения с <<41>> на <<32>>.
            \end{itemize}
        \end{minipage}
        & % Тестовый случай пройден?
        Нет.\linebreak
        Баланс счёта транзакции после обновления равен <<50>>
        \\
        & & & \\
        %\hline
        % КОНЕЦ ТЕСТА

        % Тестовый случай
        \testnumber. Изменение суммы транзакции
        & % Предусловия
        \begin{minipage}[t]{1\linewidth}
            \begin{itemize}
                \item Mock-объект для TransactionStorage c поведением по умолчанию.
                \item Mock-объект для WalletStorage c поведением по умолчанию.
                \item Реализация TransactionService c зависимостями на созданные Mock-объекты.
                \item Объект транзакции с идентификатором <<12>> на сумму <<9>> и категорией типа <<EXPENSE>>.
                \item Сумма транзакции после обновления равна <<49>>.
                \item Счёт транзакции до обновления с балансом <<55>>.
            \end{itemize}
        \end{minipage}
        & % Ожидаемые результаты
        \begin{minipage}[t]{1\linewidth}
            \begin{itemize}
                \item Был вызван метод изменения суммы транзакции с идентификатором <<12>>.
                \item Вызван метод обновления баланса счёта с <<55>> на <<15>>.
            \end{itemize}
        \end{minipage}
        & % Тестовый случай пройден?
        Да
        \\
        & & & \\
        \hline
        % КОНЕЦ ТЕСТА

        % Тестовый случай
        \testnumber. Изменение категории транзакции с изменением типа
        & % Предусловия
        \begin{minipage}[t]{1\linewidth}
            \begin{itemize}
                \item Mock-объект для TransactionStorage c поведением по умолчанию.
                \item Mock-объект для WalletStorage c поведением по умолчанию.
                \item Реализация TransactionService c зависимостями на созданные Mock-объекты.
                \item Объект транзакции с идентификатором <<12>> на сумму <<11>> и категорией типа <<EXPENSE>>.
                \item Новая категория типа <<INCOME>>.
                \item Счёт транзакции до обновления с балансом <<68>>.
            \end{itemize}
        \end{minipage}
        & % Ожидаемые результаты
        \begin{minipage}[t]{1\linewidth}
            \begin{itemize}
                \item Был вызван метод изменения категории транзакции с идентификатором <<12>>.
                \item Вызван метод обновления баланса счёта с <<68>> на <<90>>.
            \end{itemize}
        \end{minipage}
        & % Тестовый случай пройден?
        Да
        \\
        & & & \\
        \hline
        % КОНЕЦ ТЕСТА

    \end{longtable}
\end{landscape}