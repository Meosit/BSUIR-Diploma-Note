\sectioncentered*{Определения и сокращения}
\label{sec:definitions}

В настоящей пояснительной записке применяются следующие определения и сокращения.
\\

\emph{Спецификация} -- документ, который желательно полно, точно и верифицируемо определяет требования, дизайн, поведение или другие характеристики компонента или системы, и, часто, инструкции для контроля выполнения этих требований~\cite{istqb_specification}.

%\emph{Кроссплатформенность} -- способность программного обеспечения работать более чем на одной аппаратной платформе и (или) операционной системе.

%\emph{Проприетарное программное обеспечение} -- программное обеспечение, являющееся частной собственностью авторов или правообладателей и не удовлетворяющее критериям свободного ПО: свобода запуска программы в любых целях, свобода адаптации программы для любых нужд, свобода распространения, свобода улучшений исходных кодов и публикации улучшений~\cite{free_software}.

\emph{Мультипарадигмовость} -- способность языка программирования поддерживать больше чем одну парадигму (стиль) программирования.
\\

ВУЗ -- высшее учебное заведение.

ПС -- программное средство.

ПО -- программное обеспечение.

БД -- база данных.

СУБД -- система управления базами данных.

ЯП -- язык программирования.

API -- application programming interface (программный интерфейс).

SDK -- software development kit (набор средств разработки).

UI -- user interface (пользовательский интерфейс).

ТЭО -- технико-экономическое обоснование.
