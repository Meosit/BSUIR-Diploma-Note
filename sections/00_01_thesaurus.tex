\sectioncentered*{Определения и сокращения}
\label{sec:definitions}

В настоящей пояснительной записке применяются следующие определения и сокращения.
\\

\emph{Спецификация} -- документ, который желательно полно, точно и верифицируемо определяет требования, дизайн, поведение или другие характеристики компонента или системы, и, часто, инструкции для контроля выполнения этих требований~\cite{istqb_specification}.

%\emph{Кроссплатформенность} -- способность программного обеспечения работать более чем на одной аппаратной платформе и (или) операционной системе.

\emph{Фреймворк} -- программное обеспечение, облегчающее разработку и объединение разных компонентов программного проекта.

\emph{Библиотека} –- сборник подпрограмм или объектов, используемых для разработки программного обеспечения.

\emph{Мультипарадигмовость} -- способность языка программирования поддерживать больше чем одну парадигму (стиль) программирования.
\\

ПС -- программное средство.

ПО -- программное обеспечение.

БД -- база данных.

СУБД -- система управления базами данных.

ЯП -- язык программирования.

API -- application programming interface (программный интерфейс).

SDK -- software development kit (набор средств разработки).

SQL – язык структурированных запросов.

UML – унифицированный язык моделирования.

UI -- user interface (пользовательский интерфейс).

ТЭО -- технико-экономическое обоснование.
