\sectioncentered*{Определения и сокращения}
\label{sec:definitions}

В настоящей пояснительной записке применяются следующие определения и сокращения.
\\

\emph{Библиотека} -- сборник подпрограмм или объектов, используемых для разработки программного обеспечения.

\emph{Бизнес-логика} -- совокупность правил, принципов, зависимостей поведения объектов предметной области программного средства.

\emph{Mock-объект} -- тип объектов, реализующих заданные аспекты моделируемого программного окружения.

\emph{Мультипарадигмальность} -- способность языка программирования поддерживать больше чем одну парадигму (стиль) программирования.

\emph{Рефакторинг} -- процесс изменения внутренней структуры программы, не затрагивающий её внешнего поведения и имеющий целью облегчить понимание её работы~\cite{frauler_refactoring}.

\emph{Спецификация} -- документ, который желательно полно, точно и верифицируемо определяет требования, дизайн, поведение или другие характеристики компонента или системы, и, часто, инструкции для контроля выполнения этих требований~\cite{istqb_specification}.

\emph{Фреймворк} -- программное обеспечение, облегчающее разработку и объединение разных компонентов программного проекта.
\\

БД -- база данных.

ПО -- программное обеспечение.

ПС -- программное средство.

СУБД -- система управления базами данных.

ТЭО -- технико-экономическое обоснование.

API -- application programming interface (программный интерфейс).

DAO -- Data Access Object, абстрактный интерфейс к какому-либо типу базы данных или механизму хранения.

MVC -- архитектурный шаблон Model-View-Controller.

MVP -- архитектурный шаблон Model-View-Presenter.

MVVM -- архитектурный шаблон Model-View-ViewModel.

SDK -- software development kit (набор средств разработки).

SQL -- язык структурированных запросов.

UI -- User Interface (пользовательский интерфейс).

UML -- унифицированный язык моделирования.
