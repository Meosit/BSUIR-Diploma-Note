\sectioncentered*{Введение}
\addcontentsline{toc}{section}{Введение}
\label{sec:introduction}

В современном мире роль денег возросла в огромной степени.
Если еще сто лет назад денежные отношения охватывали сравнительно небольшую часть населения Земли, целые регионы и континенты практически не знали денег или имели их зачаточные формы, то современный мир <<деньгизирован>> полностью и абсолютно.
Число субъектов денежных отношений в настоящее время составляет величину порядка десяти миллиардов~\cite{money_sorgin}.

Способность быть финансово грамотным человеком не только в работе, но и в повседневной жизни является достаточно важным, нужным и полезным навыком.
Практически каждый взрослый человек зарабатывает и тратит деньги, иногда экономит, иногда планирует покупки.
Но в большинстве случаев своими деньгами люди распоряжаются, как придется, руководствуясь лишь общими правилами, что нередко приводит к необходимости тратить накопления, если таковые имеются, занимать деньги или брать кредиты.

Известно, что человек значительно легче управляет тем, что хорошо себе представляет.
Чтобы управлять расходами и понимать, через какую дыру утекают деньги, требуется представлять себе структуру расходов.
Как раз в этом и призван помочь такой инструмент, как бюджет, поскольку он позволяет прогнозировать будущие затраты (как, впрочем, и доходы) на основе информации предыдущих периодов~\cite{money_under_control}.

По статистике сервиса Google AdWords, в странах СНГ ежемесячно от 2000 до 20000 поисковых запросов связаны со словосочетаниями <<личные финансы>> и <<семейный бюджет>>~\cite{google_adwords}.
Это говорит о том, что тема учёта персонального бюджета интересна людям и они в поисках решения данной проблемы.

Целью настоящего дипломного проекта является разработка программного средства, сочетающее в себе удобство и простоту использования с основными функциям учёта персонального бюджета.

В пояснительной записке к дипломному проекту излагаются детали поэтапной разработки приложения учёта персонального бюджета.
В первом разделе приведены результаты анализа литературных источников по теме дипломного проекта, рассмотрены особенности существующих систем-аналогов, выдвинуты требования к проектируемому ПС.
Во втором разделе приведено описание функциональности проектируемого ПС, представлена спецификация функциональных требований.
В третьем разделе приведены детали проектирования и конструирования ПС.
Результатом этапа конструирования является функционирующее программное средство.
В четвертом разделе представлены доказательства того, что спроектированное ПС работает в соответствии с выдвинутыми требованиями спецификации.
В пятом разделе приведены сведения по развертыванию и запуску ПС, указаны требуемые аппаратные и программные средства.
Обоснование целесообразности создания программного средства с технико-экономической точки зрения приведено в шестом разделе.
Итоги проектирования, конструирования программного средства, а также соответствующие выводы приведены в заключении.
