\subsection{Требования к проектируемому программному средству}
\label{sec:analysis:specification}

По результатам изучения предметной области, анализа литературных источников и обзора существующих систем-аналогов сформулируем требования к проектируемому программному средству.

\subsubsection{} Назначение проекта
\label{sec:analysis:specification:purpose}

Назначением проекта является разработка программного средства для платформы \andro, автоматизирующего основные задачи по учёту персонального бюджета.

\subsubsection{} Основные функции
\label{sec:analysis:specification:functions}

Программное средство должно поддерживать следующие основные фун\-к\-ции:

\begin{itemize}
    \item управление категориями расходов и доходов;
    \item управление счетами;
    \item управление транзакциями;
    \item отображение списка транзакций по дням;
    \item расчёт и отображение статистики по категориям;
    \item расчёт и отображение агрегатных значений по счетам.
\end{itemize}

\subsubsection{} Требования к входным данным
\label{sec:analysis:specification:inputs}

Входные данные для программного средства должны быть представлены в виде вводимого пользователем с помощью клавиатуры текста и выбора доступных опций пользовательского интерфейса.

Должны быть реализованы проверки вводимых данных на корректность с отображением информации об ошибках в случае их некорректности.

\subsubsection{} Требования к выходным данным
\label{sec:analysis:specification:outputs}

Выходные данные программного средства должны быть представлены посредством отображения информации с помощью различных элементов пользовательского интерфейса.

\subsubsection{} Требования к составу и параметрам технических и программных средств
\label{sec:analysis:specification:minimal_requirements}

Одним из важнейших факторов проектирования программного средства для платформы \andro является выбор минимальной версии операционной системы. 
Данный параметр определяет соотношение между охватом аудитории и доступными средствами для разработки.
Согласно статистике использования различных версий операционных систем на \andro-устройствах и учитывая тот факт, что каждая последующая версия позволяет выполнять приложения предыдущих версий, программное средство, разработанное для версии 4.4 охватывает около 95\% всех устройств на данной платформе~\cite{android_dashboard}.

Исходя из этого, а также исходя их средних технических характеристик современных мобильных устройств, проектируемое программное средство должно функционировать на устройстве со следующими минимальными характеристиками:
\begin{itemize}
    \item версия операционной системы \andro 4.4 или выше;
    \item тактовая частота процессора от 2 ГГц или более;
    \item оперативная память 2 Гб или более;
    \item диагональ дисплея 4.5 дюймов или более.
\end{itemize}

\subsubsection{} Обоснование выбора языка и сред разработки
\label{sec:analysis:specification:lang_rationale}

В качестве языка программирования был выбран язык \emph{\kt} -- лаконичный, безопасный и прагматичный язык, совместимый с языком программирования Java.
Его можно использовать практически везде, где применяется Java: для разработки серверных приложений, приложений для платформы \andro и многого другого.
\kt прекрасно работает со всеми существующими библиотекам и фреймворками, написанными на Java, не уступая последнему в производительности~\cite{kotlin_in_action}.
Данный язык был представлен как \emph{официальный} язык разработки для платформы \andro 17 мая 2017 года~\cite{kotlin_and_android}.

Основные преимущества языка \kt~\cite{kotlin_doc}:
\begin{itemize}
  \item краткость -- конструкции и возможности языка разрабатывались с целью уменьшить 
  количество кода, но при этом не уменьшая читаемости этого кода;
  \item поддержка защиты от NullPointerException на уровне языка;
  \item полная совместимость с языком программирования Java, что позволяет использовать
  весь набор библиотек и технологий, накопленных за долгое время существования Java;
  \item возможность расширять библиотеки, не изменяя их код, что позволяет дописывать необходимую 
  функциональность без нарушений каких-либо лицензий либо поиска исходных кодов уже 
  скомпилированных библиотек;
  \item мультипарадигмовость -- \kt можно писать код как в процедурном стиле, так и в 
  объектно-ориентированном, а также благодаря поддержке функций высшего порядка можно 
  писать код и в функциональном стиле. Сочетание лучших качеств у каждого из этих стилей 
  позволяет разрабатывать приложения быстро и эффективно.
\end{itemize}

Таким образом, язык программирования \kt позволяет использовать всю существующую кодовую базу языка \emph{Java} при всей своей лаконичности и простоте, что сильно повышает скорость разработки.

Основной интегрированной средой разработки была выбрана \emph{Android Studio}. Она является официальной средой разработки для платформы проектируемого программного средства~\cite{android_studio}. 

Android Studio предоставляет множество функций, которые значительно упрощают и ускоряют разработку приложений, среди которых:
\begin{itemize}
    \item расширенный редактор макетов, способность работать с UI компонентами с помощью манипуляций мышью, функция предпросмотра макета на нескольких конфигурациях экрана;
    \item сборка приложений, основанная на Gradle;
    \item различные виды сборок и генерация нескольких выполняемых файлов приложения;
    \item статический анализатор кода (Lint), позволяющий находить проблемы производительности, несовместимости версий и другое;
    \item шаблоны основных макетов и компонентов \andro.
\end{itemize}