\section{Анализ требований к программному средству и разработка функциональных требований}
\label{sec:domain}

\subsection{Варианты использования программного средства}
\label{sec:domain:use_cases}

По результатам анализа предметной области и существующих аналогов можно сделать вывод, что проектируемое программное средство должно поддерживать ряд функций для упрощения и ускорения некоторых процессов учёта персонального бюджета, ключевыми из которых являются следующие:

\begin{enumerate}
    \item управление:
    \begin{enumerate}
        \item транзакциями, что является основной функцией в проектируемом программном средстве и позволяет регистрировать все операции с деньгами в системе;
        \item счетами, это позволяет полностью манипулировать существующими источниками средств пользователя;
        \item категориями доходов и расходов, которые помогают распределить денежные затраты для последующего анализа и оптимизации расходов;
    \end{enumerate}
    \item отображение списков:
    \begin{enumerate}
        \item транзакций, структурированных по дням, что позволяет использовать просматривать и взаимодействовать с существующими транзакциями;
        \item счетов, позволяет наглядно видеть расположение и количество текущих денежных средств;
    \end{enumerate}
    \item сводные значения по счетам, такие как общая сумма по всем валютам и количество денег, которое можно потратить до следующей зарплаты, являются крайне полезной информацией для контроля личных денежных средств и избавления от лишних трат;
    \item статистика по категориям позволит определить наиболее затратные категории расходов и узнать, куда конкретно уходят деньги.
    \item использование нескольких валют, что приносит большую гибкость и повышает удобство пользования приложением.
\end{enumerate}

Диаграмма вариантов использования, разработанная с использованием нотации \uml, представлена на рисунке~\ref{fig:domain:use_cases:model}.
\begin{sidewaysfigure}
    \centering
    \includegraphics[scale=0.58]{2_1_use_case_model.png}
    \caption{Диаграмма вариантов использования ПС}
    \label{fig:domain:use_cases:model}
\end{sidewaysfigure}

Из диаграммы видно, что основные функции программного средства, а именно управление счетами, транзакциями и категориями, состоят из основных операций работы с данными, такими как создание, изменение и удаление соответствующего объекта.
Вместе с просмотром различных списков сущностей программного средства, данные варианты использования обеспечивают пользователю возможности по учёту доходов и расходов, описанные в пункте~\ref{sec:analysis:literature:tracking}.

\subsection{Разработка инфологической модели базы данных}
\label{sec:domain:db}

Концептуальное (инфологическое) проектирование — построение семантической модели предметной области, то есть информационной модели наиболее высокого уровня абстракции.
Такая модель создается без ориентации на какую-либо конкретную СУБД и модель данных.
Основными конструктивными элементами инфологических моделей являются сущности, связи между ними и их свойства (атрибуты).

Исходя из необходимости использования в проектируемом приложении базы данных, требуется разработать ее инфологическую модель.
Для ее создания возможно использовать расширение диаграммы классов \uml, предназначенное для моделирования баз данных.
Полученная диаграмма (рисунок~\ref{fig:domain:db:model}) будет являться моделью базы данных инфологического уровня~\cite{kulikov_db_workbook}.

\begin{sidewaysfigure}
    \centering
    \includegraphics[scale=0.90]{2_2_logical_model.png}
    \caption{Инфологическая модель базы данных ПС}
    \label{fig:domain:db:model}
\end{sidewaysfigure}

Одной из особенностей разработанной модели является отсутствие сущности <<Пользователь>>.
Исходя из того, что под результатом дипломного проектирования предполагается однопользовательское программное средство, разворачиваемое на мобильном устройстве -- ПС не нуждается в сохранении дополнительной информации о пользователе.
Однако, в будущем планируется ввести возможности по синхронизации данных с удаленным сервером.
В таком случае данные о пользователе, а именно данные аутентификации и авторизации, можно будет сохранить в локальном хранилище приложения.

Далее рассматривается отдельно сущность, представленная на диаграмме.

Сущность <<Валюта>> содержит информацию о денежных единицах, которая позволяет пользователю работать с несколькими валютами одновременно.
Предполагается, что атрибут <<Код>> должен представляться в общепринятом трёхсимвольном коде валюты по стандарту ISO~4217~\cite{iso_4217}.
Атрибут <<Параметры отображения>> необходим для того, чтобы отображать конкретные денежные значения в общепринятом и знакомом для каждой валюты формате. Например, суммы в долларах США обычно записываются в виде <<\$100>>, а суммы в белорусских рублях -- <<100 руб.>>.

Сущность <<Счёт>> содержит актуальную информацию о пользовательском хранилище денежных средств.
Тип счета позволяет указать на то, является ли он сберегательным или нет.
Эта информация вносит больше гибкости в процесс планирования расходов и актуальна в момент расчета лимита расходов на день с учетом даты следующего поступления денег.
Данная сущность находится в связи с сущностью <<Валюта>>, так как каждый счёт хранит денежные средства той валюты, с которой связан.

Сущность <<Категория>> представляет собой отдельное направление расходов или доходов пользователя.
Атрибут <<Тип>> определяет вид транзакций, связанных с данной категорией и определяет, был совершен доход или расход.

Сущность <<Транзакция>> является основной в проектируемом ПС и определяет разовую операцию с денежными средствами.
Экземпляры сущности содержат в себе информацию о размере денежных средств, участвующих в операции, а также дату совершения операции.
<<Транзакция>> связана с сущностью <<Категория>>, благодаря чему можно определить, является ли конкретная транзакция расходом или доходом.
Также она связана с сущностью <<Счёт>>, что определяет валюту, в которой была проведена операция.

Сущность <<Перевод между счетами>> содержит информацию о переносе денежных средств между двумя счетами.
Данная сущность имеет две связи с сущностью <<Счёт>>, определяя, откуда и куда была перемещена сумма денег.

Сущность <<Ежемесячный доход>> представляет данные о дате ожидаемого поступления денег и позволяет оценить количество денег, которые можно ежедневно тратить до такого поступления.
<<Ежемесячный доход>> связан с сущностью <<Валюта>>, так как расчёт лимита денег на день предполагается рассчитывать среди доступных денежных средств этой же валюты.

\subsection{Разработка спецификации функциональных требований}
\label{sec:domain:specification}

На основе поставленных целей, задач и требований, определённых в пункте~\ref{sec:analysis:specification}, а также разработанной функциональной модели программного средства учёта персонального бюджета, возможно представить детализацию функций проектируемого ПС.

\subsubsection{} Поддержка нескольких валют
\label{sec:domain:specification:currencies}

При реализации функции поддержки нескольких валют должны быть учтены следующие требования:

\begin{enumerate}
    \item доступ к списку всех существующих валют;
    \item конечный список валют заблаговременно определен;
    \item коды валют соответствуют стандарту ISO~4217;
    \item запись денежных сумм осуществляется с учётом особенностей каждой валюты.
\end{enumerate}

\subsubsection{} Функция управления счетами
\label{sec:domain:specification:wallets}

Функция управления счетами должна быть реализована с учетом следующих требований:

\begin{enumerate}
    \item создание счёта;
    \item изменение любого существующего счёта;
    \item изменение баланса счёта не должно влиять на другие сущности;
    \item удаление счёта;
    \item обеспечение удаления всех связанных со счётом транзакций при его удалении;
    \item денежный перевод с одного счёта на другой с изменением соответствующих значений балансов счетов;
    \item учёт возможных разных валют счетов в денежном переводе.
\end{enumerate}

\subsubsection{} Функция отображения списка счетов
\label{sec:domain:specification:wallets_list}

Реализация функции отображения списка счетов должна учитывать следующие требования:

\begin{enumerate}
    \item выбор сортировки элементов списка среди следующих:
    \begin{enumerate}
        \item алфавитный порядок по имени счёта;
        \item по порядку создания -- наиболее ранние сверху;
        \item по порядку создания -- наиболее поздние сверху;
    \end{enumerate}
    \item элементы списка содержат следующую информацию о счёте:
    \begin{enumerate}
        \item имя;
        \item тип (повседневный или сберегательный);
        \item текущий баланс;
        \item валюта;
    \end{enumerate}
    \item обновление списка при любых операциях с транзакциями и\linebreakсчетами.
\end{enumerate}

\subsubsection{} Функция отображения сводных значений по счетам
\label{sec:domain:specification:wallets_stats}

При отображении сводных значений по счетам должны быть учтены перечисленные требования:

\begin{enumerate}
    \item просмотр количества денег в день по каждой валюте, в которых пользователь указал ежемесячное поступление денег, со следующими особенностями:
    \begin{enumerate}
        \item счета сберегательного типа не участвуют в расчёте данного показателя;
        \item сумма рассчитывается в зависимости от ближайшего по времени дня поступления средств по каждой валюте;
    \end{enumerate}
    \item расчёт и отображение общего баланса всех счетов, сгруппированных по валютам;
    \item сводные значения обновляются после любых операций счетами и транзакциями.
\end{enumerate}

\subsubsection{} Функция управления категориями
\label{sec:domain:specification:categories}

Требования, которые должны быть учтены при разработке данной функции, включают в себя:

\begin{enumerate}
    \item создание категории;
    \item удаление категории с возможностью выбора одной из двух операций над зависимыми транзакциями:
    \begin{enumerate}
        \item полное удаление;
        \item перемещение на другую существующую категорию;
    \end{enumerate}
    \item стандартные категории доходов и расходов с именем <<Без\linebreakкатегории>>;
    \item невозможность удаления стандартной категории.
\end{enumerate}

\subsubsection{} Функция отображения статистики по категориям.
\label{sec:domain:specification:categories_stats}

При реализации функции должны быть учтены следующие\linebreakтребования:

\begin{enumerate}
    \item отображение статистики в виде списков всех категорий с дополнительной информацией;
    \item списки категорий различных типов (доходов и расходов) отображаются раздельно;
    \item периода расчёта статистики и валюты, по которой производится подсчёт;
    \item выбор периода расчёта статистики и валюты, по которой производится подсчёт;
    \item выбор сортировки элементов списка среди следующих:
    \begin{enumerate}
        \item алфавитный порядок по имени категории;
        \item по убыванию общей суммы всех транзакций по данной категории;
        \item по порядку создания -- наиболее ранние сверху;
        \item по порядку создания -- наиболее поздние сверху;
    \end{enumerate}
    \item элементы списка должны содержат следующую информацию с учётом заданных валюты и периода времени:
    \begin{enumerate}
        \item имя категории;
        \item количество транзакций, совершенных для данной категории;
        \item общая сумма всех транзакций, совершенных для данной категории;
        \item процентное содержание суммы всех транзакций, совершенных для данной категории, относительно других категорий;
    \end{enumerate}
    \item статистика по категориям обновляется при любых операциях с транзакциями или категориями.
\end{enumerate}

\subsubsection{} Функция управление транзакциями
\label{sec:domain:specification:transactions}

Реализация функции управления транзакциями должна учитывать следующие требования:

\begin{enumerate}
    \item создание транзакции;
    \item изменение любых значений всех существующих транзакций;
    \item удаление любой транзакции;
    \item обновление соответствующих балансов счетов при любых операциях с транзакциями.
\end{enumerate}

\subsubsection{} Функция отображения списка транзакций по дням
\label{sec:domain:specification:transactions_list}

При реализации функции отображения транзакций необходимо принимать во внимание следующие требования:

\begin{enumerate}
    \item транзакции представлены в виде двухуровневого списка, отсортированного в порядке убывания дат совершения транзакции:
    \begin{enumerate}
        \item каждый элемент списка представляет собой один календарный день;
        \item внутренние списки содержат в себе все транзакции, совершенные в соответствующий календарный день;
        \item элемент внутреннего списка представляет собой одну\linebreakтранзакцию;
    \end{enumerate}
    \item для каждой транзакции отображается следующая информация:
    \begin{enumerate}
        \item сумма транзакции с учётом записи определённой валюты;
        \item имя счёта, связанного с данной транзакцией;
        \item имя категории, в которую была совершена транзакция;
        \item тип транзакции (доход или расход);
    \end{enumerate}
    \item просмотр информации о сумме всех транзакций по каждому дню;
    \item список обновляется при любых операциях с транзакциями.
\end{enumerate}
