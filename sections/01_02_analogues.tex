\subsection{Обзор существующих прототипов и аналогов}
\label{sec:analysis:analogues}

Для решения проблемы учёта доходов и расходов разработано большое количество различных приложений, которые включают в себя:
\begin{itemize}
    \item решения для персональных компьютеров с операционной системой Windows;
    \item веб-приложения;
    \item мобильные приложения на платформах \andro и iOS\@.
\end{itemize}

Учитывая, что в рамках дипломного проектирования предполагается разработать приложение для платформы \andro, рассматриваться будут только существующие аналоги для данной платформы.

В магазине мобильных приложений для платформы \andro, Google Play, представлены сотни приложений данной тематики, однако невозможно оценить точное число таких приложений из-за того, что Google Play не предоставляет никакой общей статистики по количеству приложений.
Но даже по такому числу приложений можно сказать, что  тематика является популярной и востребованной.
В то же время можно сделать вывод, что до сих пор не появилось доминирующего приложения, каждое решает проблему по-своему.

Достаточно важным критерием среди мобильных приложений является качество и проработанность пользовательского интерфейса.

Универсальной метрики, позволяющей оценить удобство использования и качество пользовательского интерфейса, нет.
Однако, в 2014 году компания Google представила набор рекомендаций по правильному, с их точки зрения, оформлению пользовательского интерфейса под названием <<Material Design>>~\cite{google_material_news}.
Данные рекомендации были введены для того, чтобы~\cite{google_material_guidelines}:
\begin{itemize}
    \item создать единые классические принципы отображения элементов, которые будет улучшать восприятие и позволять воплотить большое множество одновременно уникальных и привычных интерфейсов;
    \item внести единый список визуальных компонентов интерфейса для того, чтобы было быстро и просто понять, как взаимодействовать с конкретным элементом, без какого либо опыта работы с конкретным приложением.
\end{itemize}

Таким образом, можно легко отличить мобильные приложения, сделанные за последние несколько лет, от приложений, сделанных более пяти лет назад.
Также можно определить удобство использования по такому признаку, как отзывы тех людей, которые используют эти приложения.

Некоторые мобильные приложения интернет-банкинга предоставляют функции по учёту персонального бюджета, основанные на анализе расходов банковских карт данного банка.
Например, приложение <<INSYNC.BY>>, разработанное для Альфа-Банка, отображает количество потраченных денег по автоматически созданным категориям, распознанным используя тип заведения, в котором была произведена оплата (рисунок~\ref{fig:analysis:analogues:alfabank}).

\begin{figure}[h]
    \centering
    \includegraphics[scale=0.30]{1_2_alfabank.png}
    \caption{Статистика расходов по категориям приложения <<INSYNC.BY>>}
    \label{fig:analysis:analogues:alfabank}
\end{figure}

У описанного решения есть некоторые достоинства:
\begin{itemize}
    \item автоматизированный подсчет и учёт расходов;
    \item для категоризации расходов не требуются действия со стороны пользователя, достаточно производить оплату банковской картой.
\end{itemize}

Однако, несмотря на эти плюсы, у такого подхода есть ряд ощутимых недостатков в виде отсутствия следующих возможностей:
\begin{itemize}
    \item учёт расходов с карт других банков: приложение привязано к конкретному банку;
    \item пользовательское определение категорий: направления расходов определяются автоматически исходя из типов заведений, в которых был произведен платеж;
    \item определение категорий доходов;
    \item контроль наличных денег: принимаются во внимание только операции с банковскими картами.
\end{itemize}

Описанный выше тип программных средств является быстрым решением, не требующим особых усилий для внедрения в повседневную жизнь, но в то же время не позволяет в полной мере вести учёт и анализ персонального бюджета.

Помимо этого для учёта персонального бюджета существуют программные решения на базе существующих приложений.
Например, в качестве платформы для таких решений может выступать приложение для обмена мгновенными сообщениями <<Telegram>>.
Для платформы разрабатывается бот -- программное средство, развернутое на удаленном сервере, которое выполняет роль виртуального собеседника, выполняя действия в ответ на определенные команды пользователя~\cite{telegram_bots}.

У бота есть ряд преимуществ по сравнению с обычным мобильным приложением:
\begin{itemize}
    \item на разработку бота требуется куда меньше времени и, соответственно, денег в силу того, что всю инфраструктуру для отображения информации и некоторые другие возможности предоставляет платформа;
    \item <<Telegram>> существует на всех основных мобильных и настольных операционных системах, что позволяет сократить расходы на создание, тестирование и распространение собственного приложения, за исключением непосредственной функциональности бота.
\end{itemize}

Тем не менее, программные средства на базе других приложений имеют следующий список недостатков:
\begin{itemize}
    \item интерфейс ограничен только текстовыми сообщениями, что не позволяет организовать быстрый и удобный доступ ко всем доступным\linebreakфункциям;
    \item работа с приложением невозможна без соединения с интернетом в силу того, что программное средство на самом деле выполняется на удалённом сервере, а не на мобильном устройстве;
    \item приложение напрямую зависит от доступности платформы, таким образом в случае отказа платформы недоступным будет и приложение.
\end{itemize}

Одно из решений для учёта персонального бюджета на базе приложения <<Telegram>> -- бот <<Greenz>> от разработчика ООО <<Простое решение>>.
К достоинствам данного решения для учёта доходов и расходов стоит отнести следующие:
\begin{itemize}
    \item установка желаемой суммы, которую пользователь не хочет превышать за месяц;
    \item поддержка человеческого языка: бот способен понимать пользовательские запросы, сделанные в виде человеческих предложений, что упрощает взаимодействие;
    \item синхронизация с таблицами Google Spreadsheets.
\end{itemize}

Из недостатков бота <<Greenz>> стоит выделить отсутствие возможности создания собственных категорий и указания валюты.

Также для рассмотрения было выбрано несколько различных мобильных приложений, имеющих более расширенные возможности для учёта персонального бюджета.

Первое из таких приложений -- <<Money Lover -- Менеджер Расходов>> от разработчика Finsify.
Приложение является продвинутым инструментом для учета личных финансов.
Приложение имеет следующие преимущества:
\begin{itemize}
    \item установка лимит затрат и, когда он будет исчерпан, будут появляться уведомления, что больше тратить нельзя;
    \item в приложении присутствует большой выбор категорий по умолчанию, как и возможность создавать свои собственные;
    \item генерирование отчётов о расходах и доходах за произвольный\linebreakпериод;
    \item управление долгами;
    \item синхронизация пользовательских данных между несколькими устройствами: компьютером, планшетом и смартфоном и так далее.
\end{itemize}

Стоит отметить, что интерфейс <<Money Lover -- Менеджер Расходов>> разработан в соответствии к концепциями Material Design.

Однако большое количество возможностей неизбежно влечёт за собой такие недостатки, как перегруженность пользовательского интерфейса. Кроме того наличие рекламных баннеров (рисунок~\ref{fig:analysis:analogues:money_lover}), которые закрывают до 15\% рабочего пространства, осложняет быструю работу с приложением.
Ещё одной особенностью приложения является необходимость регистрации в приложении для работы даже в режиме без соединения с сетью.
Также приложение достаточно плохо русифицировано, как это можно видеть в списке стандартных категорий (см. рисунок~\ref{fig:analysis:analogues:money_lover}б).

Мобильное приложение <<Транжира -- учет финансов>> от разработчика Worms Studio является ещё одним популярным инструментом для учета личных финансов.
Оно позволяет добавлять доходы и расходы, использовать основную функциональность без авторизации и составлять аналитические отчеты.
Приложение является простым в использовании, ориентированным на быстрый учёт доходов и расходов (главный экран приложения позволяет сразу ввести сумму очередной транзакции), качественно русифицировано и имеет веб-версию.
Однако данное решение не позволяет переносить остатки бюджетов, дублировать транзакции и добавлять к ним теги, геометки и фото.
Кроме этого, большинство функций, необходимых для основательного учёта персонального бюджета, недоступны для бесплатной версии приложения.
Среди таких функций:
\begin{itemize}
    \item наличие множественных счетов;
    \item управление несколькими валютами;
    \item возможности планирования бюджета по категориям.
\end{itemize}

\begin{figure}[h]
    \centering
    \includegraphics[scale=0.32]{1_2_money_lover.png}
    \caption{Некоторые экраны приложения <<Money Lover -- Менеджер Расходов>>: а) -- главный экран; б) -- меню категорий}
    \label{fig:analysis:analogues:money_lover}
\end{figure}

Отсутствие множественных счетов не позволяет контролировать различные места расположения денег, например наличные деньги и банковская карта.
К тому же нет возможности реализовать какую-либо из описанных в предыдущем разделе методик финансового планирования.

Классическим приложением по учёту персонального бюджета является приложение <<Семейный бюджет>> от разработчика maloii, которое представлено на рисунке~\ref{fig:analysis:analogues:family_budget}.
Приложение предоставляет возможности по управлению счетами и категориями генерации отчётов, а также поддерживает древовидную структуру категорий и мультивалютные счета, что помогает легче структурировать транзакции.

Транзакции в приложении разделены на две группы: доходы и расходы.
Каждая группа представлена в отдельной вкладке.
С первого взгляда такое решение достаточно очевидно -- данные виды транзакций противоположны и объединять их нелогично.
Однако с другой стороны у такого решения имеется ряд недостатков:
\begin{itemize}
    \item нет наглядного представления о всех транзакциях за день;
    \item нельзя посмотреть относительные затраты за день (общую сумму всех расходов с учётом доходов);
    \item переводы между двумя счетами нельзя однозначно отнести к какой-либо из транзакций, поэтому в приложении они отображаются сразу в двух вкладках, в расходах -- как расход на счёте-отправителе, в доходах -- как доход для счёта-получателя.
\end{itemize}

Описанные недостатки в некоторой мере понижают наглядность и удобство использования.

Также стоит отметить, что внешний вид приложения устарел и не соответствует рекомендациям Material Design.

\begin{figure}[h]
    \centering
    \includegraphics[scale=0.95]{1_2_family_budget.png}
    \caption{Главные экраны приложения <<Семейный бюджет>>:\linebreakа) -- вкладка <<Расходы>>, б) -- вкладка <<Доходы>>, в) -- вкладка <<Счета>>}
    \label{fig:analysis:analogues:family_budget}
\end{figure}


Описанные в данном разделе программные решения по учёту персонального бюджета являются лишь частью всего множества и многообразия существующих прототипов и аналогов в данной области.
Данный анализ позволит сформировать оптимальные функциональные требования для проектируемого программного средства, учитывая преимущества и недостатки других решений, чтобы занять свою область рынка.