\sectioncentered*{Заключение}
\addcontentsline{toc}{section}{Заключение}

Предметной областью данного дипломного проекта является организация и упрощение задач учёта персонального бюджета.
Был проведен поиск существующих программные средств этого рода, по его результатам сделан вывод о несуществовании единого решения существующей задачи, а также об актуальности разработки программных средств данной тематики.
Было предложено программное средство, которое должно упростить и ускорить выполнение задач учёта персонального бюджета.

На основании проведенного анализа предметной области были выдвинуты требования к программному средству.
В качестве технологий разработки были выбраны наиболее современные существующие на данный момент средства, широко применяемые в индустрии.
Спроектированное программное средство было успешно протестировано на соответствие спецификации функциональных требований.
Исходя из анализа предметной области  и существующих аналогов можно было сделать вывод о целесообразности проектирования и разработки программной системы.
Результаты, полученные в ходе выполнения технико-экономического обоснования только подтвердили данный вывод.

В результате дипломного проектирования было разработано программное средство для упрощения и ускорения выполнения задач учёта персонального бюджета.
Целевой платформой программного средства является платформа \andro, приложение поддерживает следующие функции:
\begin{enumerate}
    \item управление сущностями:
    \begin{enumerate}
        \item категорий расходов и доходов;
        \item счетов;
        \item транзакций;
    \end{enumerate}
    \item отображение списков:
    \begin{enumerate}
        \item транзакций по дням;
        \item счетов;
        \item категорий по типу;
    \end{enumerate}
    \item расчёт и отображение:
    \begin{enumerate}
        \item статистики по категориям;
        \item сводных значений по счетам;
    \end{enumerate}
    \item использование нескольких валют.
\end{enumerate}

Приложение предполагается использовать в повседневной жизни людьми, у которых есть желание или необходимость контролировать и управлять своими денежными средствами, знать основные направления затрат и иметь возможность спланировать их в будущем.

Дальнейшее улучшение программного средства планируется в следующих направлениях:
\begin{itemize}
    \item реализация серверной части приложения для поддержки синхронизации удалённой и локальной баз данных, позволяющей пользователям иметь доступ с любого нового устройства и восстанавливать данные их в случае утери или повреждения;
    \item поддержка учёта пользовательских долгов и займов;
    \item расширенная поддержка различных подходов к планированию\linebreakбюджета;
    \item различные изменения и исправления для лучшего взаимодействия пользователя с программным средством.
\end{itemize}