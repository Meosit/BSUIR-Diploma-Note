\sectioncentered*{Заключение}
\addcontentsline{toc}{section}{Заключение}

Предметной областью данного дипломного проекта является организация и упрощение задач учёта персонального бюджета.
Был проведен поиск существующих программные средств этого рода, по его результатам сделан вывод о несуществовании единого решения существующей задачи, а также об актуальности разработки программных средств данной тематики.
Было предложено программное средство, которое должно упростить и ускорить выполнение задач учёта персонального бюджета.

На основании проведенного анализа предметной области были выдвинуты требования к программному средству.
В качестве технологий разработки были выбраны наиболее современные существующие на данный момент средства, широко применяемые в индустрии.
Спроектированное программное средство было успешно протестировано на соответствие спецификации функциональных требований.
Исходя из анализа предметной области  и существующих аналогов можно было сделать вывод о целесообразности проектирования и разработки программной системы.
Результаты, полученные в ходе выполнения технико-экономического обоснования только подтвердили данный вывод.

Разработано программное средство, которое поддерживает следующие функции:
\begin{enumerate}
    \item управление сущностями:
    \begin{enumerate}
        \item категорий расходов и доходов;
        \item счетов;
        \item транзакций;
    \end{enumerate}
    \item отображение списков:
    \begin{enumerate}
        \item транзакций по дням;
        \item счетов;
        \item категорий по типу;
    \end{enumerate}
    \item расчёт и отображение:
    \begin{enumerate}
        \item статистики по категориям;
        \item сводных значений по счетам;
    \end{enumerate}
    \item использование нескольких валют.
\end{enumerate}

Целевой платформой разработанного программного средства является платформа \andro.

Приложение предполагается использовать в повседневной жизни людьми, у которых есть желание или необходимость контролировать и управлять своими денежными средствами.

Следующая основная цель -- реализация серверной части приложения и поддержки синхронизации удалённой и локальной баз данных, позволяющей пользователям иметь доступ с любого нового устройства и восстанавливать данные их в случае утери или повреждения.
Также планируется ввести поддержку учёта пользовательских долгов и займов, расширенную поддержку различных подходов к планированию бюджета.

% TODO: Fill to 30% on the page