\sectioncentered*{Заключение}
\addcontentsline{toc}{section}{Заключение}

Предметной областью данного дипломного проекта является автоматизация и композиция некоторых задач учёта персонального бюджета.
Был проведен поиск существующих программные средств этого рода, по его результатам сделан вывод о несуществовании единого решения существующей задачи, а также об актуальности разработки программных средств данной тематики.
Было предложено программное средство, которое должно автоматизировать и упростить задачи учёта персонального бюджета.

На основании проведенного анализа предметной области были выдвинуты требования к программному средству.
В качестве технологий разработки были выбраны наиболее современные существующие на данный момент средства, широко применяемые в индустрии.
Спроектированное программное средство было успешно протестировано на соответствие спецификации функциональных требований.
Исходя из анализа предметной области и факта несуществования полных аналогов можно было сделать вывод о целесообразности проектирования и разработки программной системы.
Результаты, полученные в ходе выполнения технико-экономического обоснования только подтвердили данный вывод.

Разработано программное средство, которое поддерживает следующие функции:
\begin{itemize}
    \item поддержка множественных валют;
    \item управление категориями расходов и доходов;
    \item управление счетами;
    \item управление транзакциями;
    \item отображение списка транзакций по дням;
    \item отображение списка счетов;
    \item расчёт и отображение статистики по категориям;
    \item расчёт и отображение сводных по счетам.
\end{itemize}

ПС реализовано для платформы \andro.

Следующая основная цель -- реализация серверной части приложения и поддержки синхронизации удалённой и локальной баз данных, позволяющей пользователям иметь доступ с любого нового устройства и восстанавливать данные их в случае утери или повреждения.
Также планируется ввести поддержку учёта пользовательских долгов и займов, расширенную поддержку различных подходов к планированию бюджета.