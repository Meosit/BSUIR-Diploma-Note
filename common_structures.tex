%%%%%%%%%%%%%%%%%%%%%%%%%%%%
%       Unordered List
%%%%%%%%%%%%%%%%%%%%%%%%%%%%
\begin{itemize}
    \item пункт списка 1;
    \item пункт списка 2;
    \item пункт списка 3;
    \item пункт списка 4;
    \item пункт списка 5.
\end{itemize}


%%%%%%%%%%%%%%%%%%%%%%%%%%%%
%        Ordered List
%%%%%%%%%%%%%%%%%%%%%%%%%%%%
\begin{enumerate}
    \item Инициатором является пользователь, при этом ему необходимо предоставить адрес электронной почты и пароль, заданные при регистрации.
    \item Должна быть реализована возможность повторной аутентификации пользователя без необходимости ввода какой-либо информации.
    \item Должна быть реализована возможность восстановления пароля:
    \begin{enumerate}
        \item Для восстановления пароля пользователь должен предоставить адрес электронной почты, зарегистрированный в системе.
        \item На предоставленный адрес высылается уникальная ссылка.
        \item После перехода пользователем по данной ссылке ему предоставляется возможность установить новый пароль.
    \end{enumerate}
\end{enumerate}


%%%%%%%%%%%%%%%%%%%%%%%%%%%%
%       Centered Section without number aka ВВЕДЕНИЕ
%%%%%%%%%%%%%%%%%%%%%%%%%%%%
\sectioncentered*{Определения и сокращения}
\label{sec:definitions}


%%%%%%%%%%%%%%%%%%%%%%%%%%%%
%       Section aka 1
%%%%%%%%%%%%%%%%%%%%%%%%%%%%
\section{Анализ литературных источников, прототипов и формирование требований к проектируемому программному средству}
\label{sec:analysis}


%%%%%%%%%%%%%%%%%%%%%%%%%%%%
%     Subsection aka 1.2
%%%%%%%%%%%%%%%%%%%%%%%%%%%%
\subsection{Требования к проектируемому программному средству}
\label{sec:analysis:specification}


%%%%%%%%%%%%%%%%%%%%%%%%%%%%
%  Subsubsection aka 1.2.3
%%%%%%%%%%%%%%%%%%%%%%%%%%%%
\subsubsection{} Назначение проекта
\label{sec:analysis:specification:purpose}


%%%%%%%%%%%%%%%%%%%%%%%%%%%%
%         Figure
%%%%%%%%%%%%%%%%%%%%%%%%%%%%
\begin{figure}[H]
    % \ContinuedFloat % For multipage images
    \centering
    \includegraphics[scale=0.40]{my_figure.png}
    \caption{Некоторые экраны приложения <<Money Lover - Менеджер Расходов>>: а) -- главный экран, б) -- меню категорий}
    \label{fig:analysis:analogues:money_lover}
\end{figure}


%%%%%%%%%%%%%%%%%%%%%%%%%%%%
%      General expressions
%%%%%%%%%%%%%%%%%%%%%%%%%%%%
Данная диаграмма представлена на рисунке~\ref{fig:my:figure}.
Предложение со ссылкой.~\cite{bib_name}
\emph{Text in Italic}
Длин\-но\-сло\-во
\linebreak
<<Что-нибудь в кавычках>>
Тире -- это так.
Неразрывные~пробелы~в~предолжении.


%%%%%%%%%%%%%%%%%%%%%%%%%%%%
%      Long table
%%%%%%%%%%%%%%%%%%%%%%%%%%%%
\begin{center}
    \begin{longtable}{
    | >{\centering}m{0.5\textwidth}
    | >{\centering\arraybackslash}m{0.5\textwidth}|}
\caption{Основные классы и методы пакета <<service>>}
\label{tbl:design:database:reference}\\
\hline Класс & Описание\\\hline \endfirsthead
\caption*{Продолжение таблицы~\ref{tbl:design:database:reference}}\\\hline
\centering 1 & \centering\arraybackslash 2 \\\hline \endhead
    \hline
    \endfoot



    \end{longtable}
\end{center}